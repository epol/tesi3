\documentclass[a4paper,10pt]{article}

\usepackage{amsmath}
\usepackage{amssymb}
\usepackage{amsthm}
\usepackage{xfrac}
\usepackage[all]{xy}
\usepackage{graphicx}
%\usepackage{fullpage}
\usepackage{hyperref}
\usepackage[utf8x]{inputenc}
\usepackage[italian]{babel}

%\setlength{\parindent}{0in}

\newcounter{counter1}

\theoremstyle{plain}
\newtheorem{myteo}[counter1]{Teorema}
\newtheorem{mylem}[counter1]{Lemma}
\newtheorem{mypro}[counter1]{Proposizione}
\newtheorem{mycor}[counter1]{Corollario}
\newtheorem*{myteo*}{Teorema}
\newtheorem*{mylem*}{Lemma}
\newtheorem*{mypro*}{Proposizione}
\newtheorem*{mycor*}{Corollario}

\theoremstyle{definition}
\newtheorem{mydef}[counter1]{Definition}
\newtheorem{myes}[counter1]{Esempio}
\newtheorem{myex}[counter1]{Esercizio}
\newtheorem*{mydef*}{Definition}
\newtheorem*{myes*}{Esempio}
\newtheorem*{myex*}{Esercizio}

\theoremstyle{remark}
\newtheorem{mynot}[counter1]{Nota}
\newtheorem{myoss}[counter1]{Osservazione}
\newtheorem*{mynot*}{Nota}
\newtheorem*{myoss*}{Osservazione}


\newcommand{\obar}[1]{\overline{#1}}
\newcommand{\ubar}[1]{\underline{#1}}

\newcommand{\set}[1]{\left\{#1\right\}}
\newcommand{\pa}[1]{\left(#1\right)}
\newcommand{\ang}[1]{\left<#1\right>}
\newcommand{\bra}[1]{\left[#1\right]}
\newcommand{\abs}[1]{\left|#1\right|}
\newcommand{\norm}[1]{\left\|#1\right\|}

\newcommand{\pfrac}[2]{\pa{\frac{#1}{#2}}}
\newcommand{\bfrac}[2]{\bra{\frac{#1}{#2}}}
\newcommand{\psfrac}[2]{\pa{\sfrac{#1}{#2}}}
\newcommand{\bsfrac}[2]{\bra{\sfrac{#1}{#2}}}

\newcommand{\der}[2]{\frac{\partial #1}{\partial #2}}
\newcommand{\pder}[2]{\pfrac{\partial #1}{\partial #2}}
\newcommand{\sder}[2]{\sfrac{\partial #1}{\partial #2}}
\newcommand{\psder}[2]{\psfrac{\partial #1}{\partial #2}}

\newcommand{\intl}{\int \limits}

\DeclareMathOperator{\de}{d}
\DeclareMathOperator{\id}{Id}
\DeclareMathOperator{\len}{len}

\DeclareMathOperator{\gl}{GL}
\DeclareMathOperator{\aff}{Aff}
\DeclareMathOperator{\isom}{Isom}

\DeclareMathOperator{\im}{Im}




\title{Appunti slide}
\author{Enrico Polesel}
\date{\today}

\begin{document}

\section{Grafi e cammini minimi}

\begin{itemize}
\item Ipotesi: archi non negativi
\item Dijksta: \textit{E. W. Dijkstra. A note on two problems in connexion
  with graphs. NUMERISCHE MATHEMATIK, 1(1):269-271, 1959}
\item Prodotto minplus \textit{Noga Alon, Zvi Galil, and Oded Margalit. On the
  exponent of the all pairs shortest path problem. Journal of Computer
  and System Sciences, 54(2):255-262, 1997}
\end{itemize}

\section{Oracoli per distanze}

\begin{itemize}
\item oracolo estatto: \textit{Paolo Ferragina, Igor Nitto, and
    Rossano Venturini. On compact representations of all-
    pairs-shortest-path-distance matrices. Theoretical Computer
    Science, 411(34-36):3293-3300, 2010.  }
\end{itemize}

\newpage
\cite{funnymult,apspfmp,kshortest,kspheur,graphspanners,appdistoracl,appdistoraclplus,compactdistance,separator,cormen,dijkstra,distanceintersection}

\bibliographystyle{alpha}
\bibliography{funnymult,apspfmp,kshortest,kspheur,graphspanners,appdistoracl,appdistoraclplus,compactdistance,separator,cormen,dijkstra,distanceintersection}


\end{document}

